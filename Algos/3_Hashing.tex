\documentclass[fleqn, 12pt]{article}
\usepackage[left=0.5in, right=0.5in, top=1in, bottom=0.5in]{geometry}
\usepackage{mathexam}
\usepackage{mathtext} 				% русские буквы в фомулах
\usepackage[T2A]{fontenc}			% кодировка
\usepackage[utf8]{inputenc}			% кодировка исходного текста
\usepackage[english,russian]{babel}	% локализация и переносы
\usepackage{enumerate}
\usepackage{tabularx}
%%% Дополнительная работа с математикой
\usepackage{amsmath,amsfonts,amssymb,amsthm,mathtools,amsthm}
\usepackage{icomma} % "Умная" запятая: $0,2$ --- число, $0, 2$ --- перечисление
\usepackage{graphicx}
\usepackage{listings}
\usepackage{float}
\usepackage{enumitem}
\usepackage{indentfirst}
\relpenalty=9999
\binoppenalty=9999
\newcommand{\bigslant}[2]{{\raisebox{.2em}{$#1$}\left/\raisebox{-.2em}{$#2$}\right.}}
%чудоскобки
\newcommand{\lr}[1]{\left({#1}\right)}
\newcommand{\lrl}[1]{\left\{{#1}\right\}} 
\newcommand{\lrh}[1]{\left[{#1}\right]}
%мелкие полезности
\newcommand{\eps}{\varepsilon} 
\newcommand{\ar}{\ \ \Leftrightarrow \ \ }
\newcommand{\arr}{\ \Rightarrow \ }
\newcommand{\arl}{\ \Leftarrow \ }
%многострочная формула с выравниванием по амперсантам
\newcommand{\Split}[1]{%
	\begin{equation}
	\begin{split}
	#1
	\notag
	\end{split}
	\end{equation}
}
%системы уравнений
\newcommand{\dand}[2]{\left\{\begin{aligned}&#1\\&#2\\\end{aligned}\right.}
\newcommand{\dor}[2]{\left[\begin{aligned}&#1\\&#2\\\end{aligned}\right.}
\newcommand{\ans}[1]{\left\{\begin{aligned}#1\end{aligned}\right.}
\newcommand{\ors}[1]{\left[\begin{aligned}#1\end{aligned}\right.}	
\newcommand{\bigo}{\mathcal{O}}
\newcommand{\mb}[1]{\mathbb{#1}}
%Оформление кода
\let\ds\displaystyle
\lstset{frame=tb,
	language=C++,
	aboveskip=3mm,
	belowskip=3mm,
	showstringspaces=false,
	columns=flexible,
	basicstyle={\small\ttfamily},
	mathescape=true,
	numbers=none,
	breaklines=true,
	breakatwhitespace=true,
	tabsize=3
}
\newenvironment{myindentpar}[1]%
{\begin{list}{}%
		{\setlength{\leftmargin}{#1}}%
		\item[]
	}
	{\end{list}}
\newcommand{\image}[2] { \begin{figure}[h!] \centering	\includegraphics[scale=#2]{#1}  \end{figure} }

\begin{document}

\section{Хэширование}

\subsection{АТД <<множество>>}

\begin{enumerate}
	\item $ find(k) $
	
	\item $ insert(k) $
	
	\item $ erase(k) $
	
\end{enumerate}

Реализации:

\begin{enumerate}
	\item <<Прямая адресация>>
	
	Заводим массив размера $ |K| $. Тратим $ \bigo(1) $ на любой запрос, но приходится потратить $ \bigo(|K|) $ памяти, что часто непозволительно много (для хранения интов придется выделить массив размера $ 2^32 $ байт).
	
	\item Дерево поиска
	
	$ \bigo(\log n) $ на любой запрос и $ \bigo(n) $ памяти. Однако необходимо, чтобы было отношение порядка на объектах, что не самое слабое ограничение.
	
	\item Hash-таблицы
	
	$ \bigo(1) $ на запрос в среднем, честная $ \bigo(n) $ памяти, не требует наличия отношения порядка (но требует наличия hash-функции от объектов, что более слабое ограничение). Единственный недостаток -- амортизированная единица в асимптотике на запрос, которая может легко выродиться во что-нибудь отстойно-линейное, если написать хеширование коряво.
	
\end{enumerate}



\subsection{Хеш-таблица}

Пусть есть функция $ h \colon K \to \{ 0, \ldots, m - 1 \} $. Заведем массив размера $ m $. Храним элемент $ k $ в ячейке $ h(k) $. При заполнении массива хорошо бы вырастить его в несколько раз и перехешировать все элементы.

<рис. 1 с процессом хеширования и коллизиями>

\underline{Проблема:} различные $ k_1, k_2 $ могут отхешироваться в одинаковое $ h(k_1) = h(k_2) $. Это называется коллизией, и их нужно уметь разрешать. Но чтобы понять, как это делать (и нужно ли вообще), нужно сначала проанализировать свойства хеш-функций.

\subsubsection{Виды хеш-функций}

\begin{enumerate}
	\item $ K \subseteq \mb{Z} $
	
	$ h(k) \equiv k \mod m $
	
	или
	
	$ h(k) = [m \{ kA \}] mod m $, где $ A $ -- в принципе любое число (хотелось бы конечно вещественное, иначе неинтересно), но доказано, что хорошим является $ A = \frac{1 + \sqrt{5}}{2} $
	
	\item $ K \subseteq [0; 1] $
	
	$ h(k) =  \lfloor k \cdot c \rfloor $, где $ c $ -- некоторое число
	
	\item Строки
	
	Всенародно известный полиномиальный хеш. Пусть есть строка $ s = s_0 s_1 \ldots s_n $. Тогда определим $ h(s) = s_0 \cdot x^n + s_1 \cdot x^{n - 1} + \ldots + s_n x^0 \mod m $, где $ gcd(x, m) = 1 $. 
	
	В чем замечательность такого хеша? Ну, во-первых, мы умеем быстро (за $ \bigo(n) $ ) считать такой хеш по схеме Горнера:
	
	$ h(s_0 \ldots s_{i + 1}) = h (s_0 \ldots s_i) \cdot x + h(s_{i + 1}) $.
	
	Как естественное следствие, получаем алгоритм \textbf{Рабина-Карпа}:
	
	\begin{myindentpar}{0.5cm}
		Можно найти подстроку $ P $ в строке $ T $ за время $ \bigo( |P| + |T| ) $ при условии наличия <<хорошего>> полиномиального хеширования. 
		
		По тексту $ T $ длины $ n $ построим массив $ H \colon H[i] = h(T[i] \ldots T[i + |P|]) $. Осталось пройтись по массиву и найти хеш, который соответствует хешу шаблона: $ H[i] = h(p) $. Но вот вопрос -- как строить такой массив эффективно?
		
		Давайте запишем $ H[i] = \lr{H[i - 1] - h(T[i - 1]) \cdot x^{|P|} }x + h[i + |P|] $. Вот и научились строить!
	\end{myindentpar}
	
\end{enumerate}


\subsubsection{Какая хеш-функция считается хорошей?}

\begin{itemize}
	\item Вероятность коллизий не более чем $ \frac{1}{m} $. Однако есть проблема: мы не знаем распределения на $ K $. 
\end{itemize}

Предположим, что $ h(k) $ равновероятно на $ \{ 0, \ldots, m - 1 \} $. Тогда нам не важно распределение ключей. 


\subsubsection{Разрешение коллизий}

\begin{enumerate}
	\item Метод цепочек
	
	<рис. 2 с методом цепочек>.
	
	Храним в ячейках списки, ищем/вставляем очевидно.
	
	\underline{Утв:} Поиск отсутствующего элемента занимает $ \bigo(1 + \alpha) $, где $ \alpha = \frac{n}{m}  $ -- коэффициент заполнения (здесь и далее все утв. делаются в предположении верности гипотезы равномерного хеширования)
	
	$ E(\#\text{операций}) = 1 + E[\text{длина цепочки}] = 1 + \frac{n}{m} = \bigo(1 + \alpha) $
	
	\underline{Утв:} Успешний поиск элемента в таблице в среднем занимает $ \bigo(1 + \alpha) $. 
	
	Не можем применить гипотезу равномерного распределения к элементу, который мы ищем, т.к. он уже в таблице присутствует. 
	
	Замечание: поиск любого элемента требует столько же операций, сколько мы потратили на его добавление.
	
	$ \displaystyle E[\text{успешный поиск}] = \frac{1}{n} \cdot \sum \limits_{i = 1}^{n} E[\#\text{операций insert}] = \frac {1}{n} \cdot \sum \limits_{i = 1}{n} (1 + \frac{i - 1}{m}) = 1 + \frac{1}{nm} \cdot \sum \limits_{i = 1}^{n}(i - 1) = 1 + \frac{1}{nm} \cdot \frac{(n - 1) \cdot n)}{2} = 1 + \frac{n}{2m} - \frac{1}{2m} = \bigo(1 + \alpha)$
		
	\item Метод последовательных проб.
	
	Линейные пробы. 
	
	<рис. 1 с линейными пробами>
	
\end{enumerate}

\end{document}