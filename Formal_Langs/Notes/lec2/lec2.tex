\section{Недетерминированные конечные автоматы с $ \epsilon $--переходами}

Пример $ L $ -- слова, в которых последние 10 символов начинаются с нуля.

%\image{image6.png}{0.5}

Будем говорить, что $ NFA$-$\epsilon$ принимает слово $ x $, если существует хотя бы один путь, переводящий $ A $ в терминальное состояние.

Следующий пример:

$ L $ -- слова, в которых последние 10 символов начинаются с нуля, или слова четное длины.

%\image{img7.png}{0.5}

То есть у нас есть некоторый оракул, который нам скажет, по какому из недетерминированных переходов перейти, чтобы победить. 

\underline{Опр.} $ \epsilon $-НКА $ A = (\Sigma, Q, Q_0, T, \Delta) $ это пятерка:

\begin{itemize}
	\item $ \Sigma $ -- алфавит
	
	\item $ Q $ -- множество состояний
	
	\item $q_0$ -- начальное состояние
	
	\item $ T \subseteq F $ -- терминальные состояния
	
	\item $ \Delta \colon Q \times (\Sigma \cup \{ \epsilon \}) \to 2^\{Q\} $ -- функция переходов. Т.е. в отличии от ДФА, при переходе по символу из некоторого состояния, у нас есть целое множество вариантов (недетерминированность!)ы
\end{itemize}

\underline{Опр.} $ \epsilon $-замыканием $ E(q) $ состояния $ q \in Q $ называется множество состояний, достижимых из $ q $ только по $ \epsilon $-переходам.

\underline{Опр.} $ \epsilon $-НКА принимает слово $ a_1 \ldots a_n $, если $ \exists $ последовательность состояний $ r_0 \ldots r_n $ такая, что $ r_0 = E(q_0) $, $ \forall i \in [1; n] \lr{r_i \in E(\Delta(r_{i - 1}, a_i))} $, $ r_n \in T $

\subsection{Алгоритм детерминизации $ \epsilon $-НКА}

Пусть есть $ \epsilon $-НКА $ A = (\Sigma, Q, q_0, T, \Delta) $. Построим ДКА 

$$ B = \lr{\Sigma, 2^{Q}, E(q_0), \{ R \colon R \cap T \neq 0 \}, \delta(R, a) = \bigcup \limits_{r \in R} E\lr{\Delta\lr{r, a}}} $$

\underline{Замечание.} Разумно включать в автомат $ B $ только состояния, достижимые из начального.

\subsection{Академические регулярные выражения}

$ \emptyset $ -- $ \emptyset $
$ \epsilon $ -- $ \{ \epsilon \} $
$ a $ -- $ \{ a \} $, $ \forall a \in \Sigma $
$ (R) $ -- $ L(R) $
$ R^* $ -- $ L(R)^* $, замыкание Клини, $ L^* = \{ w_1 \ldots w_n \colon n \geqslant, \forall w_i \in L \} $
$ R_1 R_2 $ -- $ \{ ab \colon a \in L(R_1) \land b \in L(R_2) \} $
$ R_1 \mid R_2 $ -- $ L(R_1) \cup L(R_2) $
