\underline{Литература}. \textit{Хопкрофт, Мотвани, Ульман.} <<Введение в теорию языков, автоматов и грамматик>>.

Мы будем работать над \underline{конечным} алфавитом $ \Sigma $.

\textit{Слово} -- \underline{конечная} последовательность элементов алфавита.

\textit{Язык} -- множество слов (в том числе $ \epsilon $ -- пустое слово).

Примеры:

\begin{enumerate}
	\item Русский язык -- бесконечный! (т.к. есть прапрапра...внуки).
	
	\item $ Java $.
	
\end{enumerate} 

Проблема -- как проверить принадлежность слова языку? (т.е. как проверить, что набор каких-то символов является корректной программой на Java?). Для этого нужно проверить \textit{семантику} языка. Данный же курс будет практически полностью изучать другую сторону -- \textit{синтаксис} языка, закрывая глаза на семантику. 

Заметим, что вопрос о принадлежности слова языку можно сформулировать как так называемый <<Yes/No question>>. Поэтому в таком контексте $ Язык = Задача $. Например, можно сказать, что есть задача с ответом да/нет -- простое ли число? Но можно и сказать, что есть некоторый язык простых чисел, и эти две вещи сколько-то равносильны друг другу.

Давайте подсчитаем количество языков. Множество слов над $ \Sigma $ -- счетное. Множество языков над алфавитом $ \Sigma = 2^{\mathcal{N}}$ -- несчетное, но вот множество конечных языков над $ \Sigma $ -- счетно. 

Таким образом, множество программ на $ Java $ (множество слов над конечных языком) -- счетно. А вот задач да/нет -- несчетно. Из этого следует много интересных следствий, в частности, что задач с ответом да/нет меньше, чем программ $ Java $ (более того, их ничтожно мало). Или что невозможно для любого действительного числа написать программу, которая бы бесконечно выводила его знаки с все более увеличивающейся точностью. А чисел, для которых такое возможно, также ничтожно мало (такие числа называются вычислимыми)

\underline{Опр.} \textit{Вычислимым} называется язык, принадлежность слова которому можно проверить некоторой программой. 

Мысль: будем изучать еще более узкий класс языков, нежели вычислимые потому, что с вычислимыми языками есть проблема. Вот пусть мы его придумали, и сказали, что якобы придумали программу, которая проверяет принадлежность этому языку. Но про такие программы очень тяжело что-то доказать! (проблемы останова, все такое)

